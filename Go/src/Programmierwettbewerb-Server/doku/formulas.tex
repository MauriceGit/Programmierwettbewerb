\documentclass[10pt,a4paper]{article}
\usepackage[latin1]{inputenc}
\usepackage{amsmath}
\usepackage{amsfonts}
\usepackage{amssymb}
\usepackage{graphicx}
\author{Maurice Tollmien}
\title{Formulas for the programming Contest 2016}
\begin{document}

\thispagestyle{empty}
\begin{center}
\textbf{\large{Some principles and formulas we use in the Agar-Server, that you use in your bots logic}}
\end{center}

\section{General Information}
\begin{itemize}
\item The server runs with consistently 30fps (So $\delta t = 0.03$). \newline
      When calculations take too long and 30fps are not possible, there will be less calculations per second
      but $\delta t$ will still be corrected to 0.03.
\item The calculation order is non-deterministic. That means, that there is no advantage over another player 
	  if one blob is larger or longer in play.
\item Food and Toxins are randomly added during play. It is possible, that a toxin spawns 
      inside a blob and causing it to explode (Will not happen very often!).

\end{itemize}      
      
\section{Splitting/Exploding/Throwing}
\begin{itemize}
\item When a blob explodes, a blob/toxin splits or food is thrown, the new blobs/food/toxin will be given an
      additional velocity, which will be added to the normal movement.
\item The additional velocity decreases over time. At every timestep the velocity is multiplied by $0.95$. \newline
      So: $newVelocity = oldVelocity * 0.95$ \newline
      This is identical for blobs, food and toxin.	
\item The following velocities will be added: \newline
      food:  150 \newline
      toxin: 
\end{itemize}

\section{Toxin Calculations}
\begin{itemize}
\item When a toxin is split and moving, it will adopt the velocity of the last food that is thrown into the toxin.
\end{itemize}

\section{Food Calculations}
\begin{itemize}
\item When a food is thrown, it will always move in the direction of the bots target.
\item Every blob (when exploded or split) of one bot will throw a food. This can result in multiple foods thrown 
      in one time step.
\end{itemize}

\end{document}